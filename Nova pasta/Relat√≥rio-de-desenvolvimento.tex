% Options for packages loaded elsewhere
\PassOptionsToPackage{unicode}{hyperref}
\PassOptionsToPackage{hyphens}{url}
%
\documentclass[
]{article}
\usepackage{amsmath,amssymb}
\usepackage{lmodern}
\usepackage{ifxetex,ifluatex}
\ifnum 0\ifxetex 1\fi\ifluatex 1\fi=0 % if pdftex
  \usepackage[T1]{fontenc}
  \usepackage[utf8]{inputenc}
  \usepackage{textcomp} % provide euro and other symbols
\else % if luatex or xetex
  \usepackage{unicode-math}
  \defaultfontfeatures{Scale=MatchLowercase}
  \defaultfontfeatures[\rmfamily]{Ligatures=TeX,Scale=1}
\fi
% Use upquote if available, for straight quotes in verbatim environments
\IfFileExists{upquote.sty}{\usepackage{upquote}}{}
\IfFileExists{microtype.sty}{% use microtype if available
  \usepackage[]{microtype}
  \UseMicrotypeSet[protrusion]{basicmath} % disable protrusion for tt fonts
}{}
\makeatletter
\@ifundefined{KOMAClassName}{% if non-KOMA class
  \IfFileExists{parskip.sty}{%
    \usepackage{parskip}
  }{% else
    \setlength{\parindent}{0pt}
    \setlength{\parskip}{6pt plus 2pt minus 1pt}}
}{% if KOMA class
  \KOMAoptions{parskip=half}}
\makeatother
\usepackage{xcolor}
\IfFileExists{xurl.sty}{\usepackage{xurl}}{} % add URL line breaks if available
\IfFileExists{bookmark.sty}{\usepackage{bookmark}}{\usepackage{hyperref}}
\hypersetup{
  pdftitle={Dúvida sobre implementação da ferramenta para AED e elaboração de relatório},
  pdfauthor={Fernando Bispo},
  hidelinks,
  pdfcreator={LaTeX via pandoc}}
\urlstyle{same} % disable monospaced font for URLs
\usepackage[margin=1in]{geometry}
\usepackage{color}
\usepackage{fancyvrb}
\newcommand{\VerbBar}{|}
\newcommand{\VERB}{\Verb[commandchars=\\\{\}]}
\DefineVerbatimEnvironment{Highlighting}{Verbatim}{commandchars=\\\{\}}
% Add ',fontsize=\small' for more characters per line
\usepackage{framed}
\definecolor{shadecolor}{RGB}{248,248,248}
\newenvironment{Shaded}{\begin{snugshade}}{\end{snugshade}}
\newcommand{\AlertTok}[1]{\textcolor[rgb]{0.94,0.16,0.16}{#1}}
\newcommand{\AnnotationTok}[1]{\textcolor[rgb]{0.56,0.35,0.01}{\textbf{\textit{#1}}}}
\newcommand{\AttributeTok}[1]{\textcolor[rgb]{0.77,0.63,0.00}{#1}}
\newcommand{\BaseNTok}[1]{\textcolor[rgb]{0.00,0.00,0.81}{#1}}
\newcommand{\BuiltInTok}[1]{#1}
\newcommand{\CharTok}[1]{\textcolor[rgb]{0.31,0.60,0.02}{#1}}
\newcommand{\CommentTok}[1]{\textcolor[rgb]{0.56,0.35,0.01}{\textit{#1}}}
\newcommand{\CommentVarTok}[1]{\textcolor[rgb]{0.56,0.35,0.01}{\textbf{\textit{#1}}}}
\newcommand{\ConstantTok}[1]{\textcolor[rgb]{0.00,0.00,0.00}{#1}}
\newcommand{\ControlFlowTok}[1]{\textcolor[rgb]{0.13,0.29,0.53}{\textbf{#1}}}
\newcommand{\DataTypeTok}[1]{\textcolor[rgb]{0.13,0.29,0.53}{#1}}
\newcommand{\DecValTok}[1]{\textcolor[rgb]{0.00,0.00,0.81}{#1}}
\newcommand{\DocumentationTok}[1]{\textcolor[rgb]{0.56,0.35,0.01}{\textbf{\textit{#1}}}}
\newcommand{\ErrorTok}[1]{\textcolor[rgb]{0.64,0.00,0.00}{\textbf{#1}}}
\newcommand{\ExtensionTok}[1]{#1}
\newcommand{\FloatTok}[1]{\textcolor[rgb]{0.00,0.00,0.81}{#1}}
\newcommand{\FunctionTok}[1]{\textcolor[rgb]{0.00,0.00,0.00}{#1}}
\newcommand{\ImportTok}[1]{#1}
\newcommand{\InformationTok}[1]{\textcolor[rgb]{0.56,0.35,0.01}{\textbf{\textit{#1}}}}
\newcommand{\KeywordTok}[1]{\textcolor[rgb]{0.13,0.29,0.53}{\textbf{#1}}}
\newcommand{\NormalTok}[1]{#1}
\newcommand{\OperatorTok}[1]{\textcolor[rgb]{0.81,0.36,0.00}{\textbf{#1}}}
\newcommand{\OtherTok}[1]{\textcolor[rgb]{0.56,0.35,0.01}{#1}}
\newcommand{\PreprocessorTok}[1]{\textcolor[rgb]{0.56,0.35,0.01}{\textit{#1}}}
\newcommand{\RegionMarkerTok}[1]{#1}
\newcommand{\SpecialCharTok}[1]{\textcolor[rgb]{0.00,0.00,0.00}{#1}}
\newcommand{\SpecialStringTok}[1]{\textcolor[rgb]{0.31,0.60,0.02}{#1}}
\newcommand{\StringTok}[1]{\textcolor[rgb]{0.31,0.60,0.02}{#1}}
\newcommand{\VariableTok}[1]{\textcolor[rgb]{0.00,0.00,0.00}{#1}}
\newcommand{\VerbatimStringTok}[1]{\textcolor[rgb]{0.31,0.60,0.02}{#1}}
\newcommand{\WarningTok}[1]{\textcolor[rgb]{0.56,0.35,0.01}{\textbf{\textit{#1}}}}
\usepackage{graphicx}
\makeatletter
\def\maxwidth{\ifdim\Gin@nat@width>\linewidth\linewidth\else\Gin@nat@width\fi}
\def\maxheight{\ifdim\Gin@nat@height>\textheight\textheight\else\Gin@nat@height\fi}
\makeatother
% Scale images if necessary, so that they will not overflow the page
% margins by default, and it is still possible to overwrite the defaults
% using explicit options in \includegraphics[width, height, ...]{}
\setkeys{Gin}{width=\maxwidth,height=\maxheight,keepaspectratio}
% Set default figure placement to htbp
\makeatletter
\def\fps@figure{htbp}
\makeatother
\setlength{\emergencystretch}{3em} % prevent overfull lines
\providecommand{\tightlist}{%
  \setlength{\itemsep}{0pt}\setlength{\parskip}{0pt}}
\setcounter{secnumdepth}{-\maxdimen} % remove section numbering
\ifluatex
  \usepackage{selnolig}  % disable illegal ligatures
\fi

\title{Dúvida sobre implementação da ferramenta para AED e elaboração de
relatório}
\author{Fernando Bispo}
\date{02/07/2021}

\begin{document}
\maketitle

\hypertarget{intruduuxe7uxe3o}{%
\section{Intrudução}\label{intruduuxe7uxe3o}}

No intuito de refazer os passos de elaboração do relatório confeccionado
inicialmente com ferramentas da Microsoft, estou utilizando os
conhecimentos adiquiridos até aqui com programação em R a fim de obter
proficiência na manipulação da ferramenta. Iniciei a tratativa do banco
de dados (BD) fornecido com as respostas do Questionário SABE
Disciplinas 2018 e exponho aqui meu desenvolvimento no intuito de obter
maiores orientações de como melhorar meu desempenho e auxílio para
aplicar a teoria obtida até o momento na faculdade.

\hypertarget{bicliotecas-utiizadas}{%
\subsection{Bicliotecas Utiizadas}\label{bicliotecas-utiizadas}}

\begin{Shaded}
\begin{Highlighting}[]
\FunctionTok{library}\NormalTok{(dplyr)  }
\end{Highlighting}
\end{Shaded}

\begin{verbatim}
## 
## Attaching package: 'dplyr'
\end{verbatim}

\begin{verbatim}
## The following objects are masked from 'package:stats':
## 
##     filter, lag
\end{verbatim}

\begin{verbatim}
## The following objects are masked from 'package:base':
## 
##     intersect, setdiff, setequal, union
\end{verbatim}

\begin{Shaded}
\begin{Highlighting}[]
\FunctionTok{library}\NormalTok{(ggplot2)  }
\FunctionTok{library}\NormalTok{(ggrepel)  }
\end{Highlighting}
\end{Shaded}

\hypertarget{importauxe7uxe3o-do-bd}{%
\subsection{Importação do BD}\label{importauxe7uxe3o-do-bd}}

Iniciei importando o BD para análise inicial e manipulação dos dados.

\begin{Shaded}
\begin{Highlighting}[]
\NormalTok{disciplinas\_2018\_1 }\OtherTok{\textless{}{-}}\NormalTok{ readr}\SpecialCharTok{::}\FunctionTok{read\_csv2}\NormalTok{(}\StringTok{"BD/SABE {-} Disciplinas 2018.1 e 2018.2/disciplinas\_2018\_1.csv"}\NormalTok{)  }
\end{Highlighting}
\end{Shaded}

\begin{verbatim}
## i Using '\',\'' as decimal and '\'.\'' as grouping mark. Use `read_delim()` for more control.
\end{verbatim}

\begin{verbatim}
## 
## -- Column specification --------------------------------------------------------
## cols(
##   ID = col_double(),
##   q1 = col_double(),
##   q2 = col_double(),
##   q3 = col_double(),
##   q4 = col_double(),
##   q5 = col_double(),
##   q6 = col_double(),
##   q7 = col_double(),
##   q8 = col_double(),
##   q9 = col_double(),
##   q10 = col_double(),
##   q11 = col_double(),
##   q12 = col_double(),
##   comentariodisc = col_character(),
##   MATRICULA = col_double(),
##   DISCIPLINA = col_character(),
##   TURMA = col_double(),
##   SITUACAO = col_character()
## )
\end{verbatim}

\begin{Shaded}
\begin{Highlighting}[]
\NormalTok{disciplinas\_2018\_2 }\OtherTok{\textless{}{-}}\NormalTok{ readr}\SpecialCharTok{::}\FunctionTok{read\_csv2}\NormalTok{(}\StringTok{"BD/SABE {-} Disciplinas 2018.1 e 2018.2/disciplinas\_2018\_2.csv"}\NormalTok{) }
\end{Highlighting}
\end{Shaded}

\begin{verbatim}
## i Using '\',\'' as decimal and '\'.\'' as grouping mark. Use `read_delim()` for more control.
\end{verbatim}

\begin{verbatim}
## 
## -- Column specification --------------------------------------------------------
## cols(
##   ID = col_double(),
##   q1 = col_double(),
##   q2 = col_double(),
##   q3 = col_double(),
##   q4 = col_double(),
##   q5 = col_double(),
##   q6 = col_double(),
##   q7 = col_double(),
##   q8 = col_double(),
##   q9 = col_double(),
##   q10 = col_double(),
##   q11 = col_double(),
##   q12 = col_double(),
##   comentdisc = col_character(),
##   MATRICULA = col_double(),
##   DISCIPLINA = col_character(),
##   TURMA = col_double(),
##   SITUACAO = col_character()
## )
\end{verbatim}

Após importação analisei os nomes das variáveis com o intuito de de
juntar os dois BDs e para não haver confusão entre eles criei a variável
SEMESTRE para obter essa diferenciação.

\begin{Shaded}
\begin{Highlighting}[]
\FunctionTok{colnames}\NormalTok{(disciplinas\_2018\_1)}
\end{Highlighting}
\end{Shaded}

\begin{verbatim}
##  [1] "ID"             "q1"             "q2"             "q3"            
##  [5] "q4"             "q5"             "q6"             "q7"            
##  [9] "q8"             "q9"             "q10"            "q11"           
## [13] "q12"            "comentariodisc" "MATRICULA"      "DISCIPLINA"    
## [17] "TURMA"          "SITUACAO"
\end{verbatim}

\begin{Shaded}
\begin{Highlighting}[]
\FunctionTok{colnames}\NormalTok{(disciplinas\_2018\_2)}
\end{Highlighting}
\end{Shaded}

\begin{verbatim}
##  [1] "ID"         "q1"         "q2"         "q3"         "q4"        
##  [6] "q5"         "q6"         "q7"         "q8"         "q9"        
## [11] "q10"        "q11"        "q12"        "comentdisc" "MATRICULA" 
## [16] "DISCIPLINA" "TURMA"      "SITUACAO"
\end{verbatim}

Constatei que a variavel comentários da disciplina estava com nome
diferente, então renomeei em ambos os BDs e criei a VA SEMESTRE

\begin{Shaded}
\begin{Highlighting}[]
\NormalTok{disciplinas\_2018\_1 }\OtherTok{\textless{}{-}}\NormalTok{ disciplinas\_2018\_1 }\SpecialCharTok{\%\textgreater{}\%} 
  \FunctionTok{rename}\NormalTok{(}\AttributeTok{COMENTARIOS\_DISC =}\NormalTok{ comentariodisc)}

\NormalTok{disciplinas\_2018\_2 }\OtherTok{\textless{}{-}}\NormalTok{ disciplinas\_2018\_2 }\SpecialCharTok{\%\textgreater{}\%} 
  \FunctionTok{rename}\NormalTok{(}\AttributeTok{COMENTARIOS\_DISC =}\NormalTok{ comentdisc)}

\NormalTok{disciplinas\_2018\_1 }\OtherTok{\textless{}{-}}\NormalTok{ disciplinas\_2018\_1 }\SpecialCharTok{\%\textgreater{}\%} 
  \FunctionTok{mutate}\NormalTok{(}\AttributeTok{SEMESTRE =} \FloatTok{2018.1}\NormalTok{)}

\NormalTok{disciplinas\_2018\_2 }\OtherTok{\textless{}{-}}\NormalTok{ disciplinas\_2018\_2 }\SpecialCharTok{\%\textgreater{}\%} 
  \FunctionTok{mutate}\NormalTok{(}\AttributeTok{SEMESTRE =} \FloatTok{2018.2}\NormalTok{)}
\end{Highlighting}
\end{Shaded}

Após esse procedimento uni os dois BDs

\begin{Shaded}
\begin{Highlighting}[]
\NormalTok{disciplinas\_2018 }\OtherTok{\textless{}{-}} \FunctionTok{full\_join}\NormalTok{(disciplinas\_2018\_1, disciplinas\_2018\_2)}
\end{Highlighting}
\end{Shaded}

\begin{verbatim}
## Joining, by = c("ID", "q1", "q2", "q3", "q4", "q5", "q6", "q7", "q8", "q9", "q10", "q11", "q12", "COMENTARIOS_DISC", "MATRICULA", "DISCIPLINA", "TURMA", "SITUACAO", "SEMESTRE")
\end{verbatim}

Percebi que as variáveis do BD não eram numéricas, então converti as VAs
do BD para fator e algumas poucas para caractere

\begin{Shaded}
\begin{Highlighting}[]
\NormalTok{disciplinas\_2018 }\OtherTok{\textless{}{-}}\NormalTok{ disciplinas\_2018 }\SpecialCharTok{\%\textgreater{}\%} 
  \FunctionTok{mutate\_all}\NormalTok{(as.factor)}

\CommentTok{\#Transformando algumas variaveis do BD em caractere e alterando o formato de strings.}
\NormalTok{disciplinas\_2018 }\OtherTok{\textless{}{-}}\NormalTok{ disciplinas\_2018 }\SpecialCharTok{\%\textgreater{}\%} 
  \FunctionTok{mutate}\NormalTok{(}\AttributeTok{COMENTARIOS\_DISC =} \FunctionTok{as.character}\NormalTok{(COMENTARIOS\_DISC),}
         \AttributeTok{DISCIPLINA =} \FunctionTok{as.character}\NormalTok{(DISCIPLINA),}
         \AttributeTok{SITUACAO =}\NormalTok{ stringr}\SpecialCharTok{::}\FunctionTok{str\_to\_title}\NormalTok{(SITUACAO))}

\FunctionTok{glimpse}\NormalTok{(disciplinas\_2018)}
\end{Highlighting}
\end{Shaded}

\begin{verbatim}
## Rows: 154
## Columns: 19
## $ ID               <fct> 1, 2, 3, 4, 5, 6, 7, 8, 9, 10, 11, 12, 13, 14, 15, 16~
## $ q1               <fct> 5, 5, 5, 5, 5, 2, 5, 5, 2, 5, 3, 5, 5, 4, 5, 5, 5, 5,~
## $ q2               <fct> 5, 5, 5, 5, 5, 3, 5, 5, 5, 5, 5, 2, 5, 5, 5, 5, 5, 5,~
## $ q3               <fct> 5, 5, 5, 5, 2, 4, 5, 5, 3, 5, 1, 5, 5, 5, 5, 5, 5, 5,~
## $ q4               <fct> 5, 5, 5, 5, 2, 4, 5, 5, 3, 5, 1, 4, 5, 2, 5, 3, 5, 5,~
## $ q5               <fct> 5, 5, 5, 5, 3, 1, 5, 5, 5, 5, 2, 5, 5, 4, 5, 5, 5, 5,~
## $ q6               <fct> 5, 5, 5, 5, 3, 4, 5, 5, 3, 5, 2, 4, 5, 3, 5, 4, 5, 5,~
## $ q7               <fct> 5, 5, 5, 5, 5, 4, 5, 5, 5, 5, 5, 4, 5, 5, 5, 5, 5, 5,~
## $ q8               <fct> 5, 5, 5, 5, 4, 4, 5, 5, 3, 5, 2, 5, 5, 3, 5, 2, 5, 5,~
## $ q9               <fct> 5, 5, 5, 5, 5, 2, 5, 5, 5, 5, 2, 5, 5, 5, 5, 5, 5, 5,~
## $ q10              <fct> 5, 5, 2, 5, 5, 3, 5, 5, 5, 5, 5, 5, 5, 5, 5, 5, 5, 5,~
## $ q11              <fct> 4, 5, 5, 5, 5, 2, 4, 5, 5, 5, 4, 5, 4, 5, 2, 5, 3, 1,~
## $ q12              <fct> 5, 2, 5, 5, 5, 4, 4, 5, 4, 5, 4, 4, 4, 5, 5, 5, 4, 5,~
## $ COMENTARIOS_DISC <chr> NA, NA, NA, NA, NA, NA, "Melhor professora !", NA, NA~
## $ MATRICULA        <fct> 201517019, 217220657, 201517019, 201517019, 217220657~
## $ DISCIPLINA       <chr> "ISCA82", "MAT229", "MAT045", "MAT224", "MATD43", "MA~
## $ TURMA            <fct> 90900, 10100, 20200, 10100, 10000, 10000, 10100, 1000~
## $ SITUACAO         <chr> "Aprovado", "Aprovado", "Aprovado", "Aprovado", "Apro~
## $ SEMESTRE         <fct> 2018.1, 2018.1, 2018.1, 2018.1, 2018.1, 2018.1, 2018.~
\end{verbatim}

\hypertarget{criauxe7uxe3o-de-gruxe1ficos}{%
\subsection{Criação de gráficos}\label{criauxe7uxe3o-de-gruxe1ficos}}

Tentando replicar os gráficos feitos no relatório iniciei com o das
disciplinas.

\begin{Shaded}
\begin{Highlighting}[]
\NormalTok{disciplinas\_2018 }\SpecialCharTok{\%\textgreater{}\%}
  \FunctionTok{count}\NormalTok{(DISCIPLINA) }\SpecialCharTok{\%\textgreater{}\%}
  \CommentTok{\# top\_n(10, n) \%\textgreater{}\%  Estou na dúvida se devo listar todas as disciplinas ou apenas as que possuem mais matriculados.}
  \FunctionTok{mutate}\NormalTok{(}\AttributeTok{DISCIPLINA =}\NormalTok{ forcats}\SpecialCharTok{::}\FunctionTok{fct\_reorder}\NormalTok{(DISCIPLINA, n)) }\SpecialCharTok{\%\textgreater{}\%}
  \FunctionTok{ggplot}\NormalTok{(}\FunctionTok{aes}\NormalTok{(}\AttributeTok{x =}\NormalTok{ DISCIPLINA)) }\SpecialCharTok{+}
  \FunctionTok{geom\_col}\NormalTok{(}\FunctionTok{aes}\NormalTok{(}\AttributeTok{y =}\NormalTok{ n), }\AttributeTok{show.legend =} \ConstantTok{FALSE}\NormalTok{) }\SpecialCharTok{+}
  \FunctionTok{geom\_text}\NormalTok{(}\FunctionTok{aes}\NormalTok{(}\AttributeTok{y =}\NormalTok{ n}\SpecialCharTok{/}\DecValTok{2}\NormalTok{, }\AttributeTok{label =}\NormalTok{ n)) }\SpecialCharTok{+}
  \FunctionTok{coord\_flip}\NormalTok{()}
\end{Highlighting}
\end{Shaded}

\includegraphics{Relatório-de-desenvolvimento_files/figure-latex/unnamed-chunk-7-1.pdf}

Após ler o relatório que a senhora desenvolveu, tentei replicar um dos
gráficos e então surgiu a primeira dúvida: Como inserir os valores das
respectivas porcentagens no gráfico de forma a deixalos centralizados
nos seus respectivos locais?

\begin{Shaded}
\begin{Highlighting}[]
\NormalTok{disciplinas\_2018 }\SpecialCharTok{\%\textgreater{}\%}  
  \FunctionTok{group\_by}\NormalTok{(SEMESTRE) }\SpecialCharTok{\%\textgreater{}\%} 
  \FunctionTok{count}\NormalTok{(SITUACAO) }\SpecialCharTok{\%\textgreater{}\%} 
  \FunctionTok{mutate}\NormalTok{(}\AttributeTok{Fr =}\NormalTok{ n}\SpecialCharTok{/}\FunctionTok{sum}\NormalTok{(n)}\SpecialCharTok{*}\DecValTok{100}\NormalTok{) }\SpecialCharTok{\%\textgreater{}\%}
  \FunctionTok{ggplot}\NormalTok{(}\FunctionTok{aes}\NormalTok{(}\AttributeTok{x =}\NormalTok{ SEMESTRE, }\AttributeTok{y =}\NormalTok{ Fr, }\AttributeTok{fill =}\NormalTok{ SITUACAO)) }\SpecialCharTok{+}
  \FunctionTok{geom\_col}\NormalTok{() }\SpecialCharTok{+}
  \FunctionTok{geom\_text\_repel}\NormalTok{(}\FunctionTok{aes}\NormalTok{(}\AttributeTok{label =} \FunctionTok{round}\NormalTok{(Fr,}\DecValTok{0}\NormalTok{))) }\SpecialCharTok{+}
  \FunctionTok{labs}\NormalTok{(}\AttributeTok{x =} \StringTok{"Semestre"}\NormalTok{, }\AttributeTok{y =} \StringTok{"Freqência (\%)"}\NormalTok{)}
\end{Highlighting}
\end{Shaded}

\includegraphics{Relatório-de-desenvolvimento_files/figure-latex/unnamed-chunk-8-1.pdf}

\end{document}
